%%%%%%%%%%%%%%%%%%%%%%%%%%%%%%%%%%%%%%%%%%%%%%%%%%%%%%%%%%%%%%%%%%%%%%%%%%%%%
%%%                                                                       %%%
%%%            LaTeX MACRO FOR COOLNEWS (Journal Abstracts)               %%%
%%%                                                                       %%%
%%%    Simply fill in the information between the brackets { } below      %%%
%%%    and e-mail to coolnews@jila.colorado.edu.  Try not to use          %%%
%%%    specially-defined Latex commands, but if you must then include     %%% 
%%%    their definitions.                                                 %%%
%%%                                                                       %%%
%%%%%%%%%%%%%%%%%%%%%%%%%%%%%%%%%%%%%%%%%%%%%%%%%%%%%%%%%%%%%%%%%%%%%%%%%%%%%

%% To run Latex on your journal abstract, first remove the % sign in the
%% nine lines below, and in the \end{document} line.

\documentstyle{article}
\textwidth 18cm
\textheight 23cm
\oddsidemargin -1cm
\topmargin 0cm
\parskip 0.15cm
\parindent 0pt
\small
\begin{document}

%%            ------ TITLE ---
%% Write the title of your paper between the brackets. Please
%% capitalize only the first letter of each word.
{\large\bf{CARMA CO(\textit{J} = $2-1$) Observations of the Circumstellar Envelope of Betelgeuse}}

%%            ------ AUTHORS -----
%% Here comes the author(s) of the paper, please indicate within $^...$ 
%% the number which corresponds to the institute of each author.
{\bf{ Eamon O'Gorman$^1$, Graham M. Harper$^1$, Joanna M. Brown$^2$, Alexander Brown$^3$, Seth Redfield$^4$, Matthew J. Richter$^5$ and Miguel A. Requena-Torres$^6$ }}

%%           ------INSTITUTIONS ---
%% Here write your institute name(s) and address(es), 
%% the number in $^..$ indicates your author number, for example:
$^1$ {School of Physics, Trinity College Dublin, Dublin 2, Ireland.} \\
$^2$ {Harvard-Smithsonian Center for Astrophysics, 60 Garden Street, MS-78, Cambridge, MA 02138, USA.} \\
$^3$ {Center for Astrophysics and Space Astronomy, University of Colorado, Boulder, CO 80309-0389, USA.} \\
$^4$ {Astronomy Department, Van Vleck Observatory, Wesleyan University, Middletown, CT 06459, USA.} \\
$^5$ {Physics Department, UC Davis, 1 Shields Avenue, Davis, CA 95616, USA.} \\
$^6$ {Max-Planck-Institut f\"ur Radioastronomie, Auf dem H\"ugel 69, 53121 Bonn, Germany.}

%%            ------ABSTRACT------
%% Enter the text of your abstract between these brackets:
{We report radio interferometric observations of the $\rm{{}^{12}}$C$\rm{{}^{16}}O$ 1.3\,mm \textit{J} = $2-1$ emission line in the circumstellar envelope of the M supergiant $\alpha$ Ori and have detected and separated both the S1 and S2 flow components for the first time. Observations were made with the Combined Array for Research in Millimeter-wave Astronomy (CARMA) interferometer in the C, D, and E antenna configurations. We obtain good $u-v$ coverage (5--280\,k$\lambda$) by combining data from all three configurations allowing us to trace spatial scales as small as $0^{\prime\prime}.9$ over a $32^{\prime\prime}$ field of view. The high spectral and spatial resolution C configuration line profile shows that the inner S1 flow has slightly asymmetric outflow velocities ranging from $-9.0\>{\rm km\>s}^{-1}$ to $+10.6\>{\rm km\>s}^{-1}$ with respect to the stellar rest frame. We find little evidence  for the outer S2 flow in this configuration because the majority of this emission has been spatially-filtered (resolved out) by the array. We also report a SOFIA-GREAT CO(\textit{J} = $12-11$) emission line profile which we associate with this inner higher excitation S1 flow. The outer S2 flow appears in the D and E configuration maps and its outflow velocity is found to be in good agreement with high resolution optical spectroscopy of K\,I obtained at the McDonald Observatory. We image both S1 and S2 in the multi-configuration maps and see a gradual change in the angular size of the emission in the high absolute velocity maps. We assign an outer radius of 4$^{\prime\prime}$ to S1 and propose that S2 extends beyond CARMA's field of view (32$^{\prime\prime}$ at 1.3 mm) out to a radius of 17$^{\prime\prime}$ which is larger than recent single-dish observations have indicated. When azimuthally averaged, the intensity fall-off for both flows is found to be proportional to \textit{R}$^{-1}$, where \textit{R} is the projected radius, indicating optically thin winds with $\rm{\rho \propto \textit{R}^{-2}}$.}

%%            -----PUBLICATION STATUS---
%% Here give the  publication status (Accepted by, Submitted to) 
%% and the name of journal, for example:
{ Accepted by AJ }

%%            -----E-MAIL ADDRESS-----
%% Here give the e-mail address for preprint requests, for example:
{{\em For preprints contact}: eogorma@tcd.ie}
%%

%%            ----FTP or WWW ADDRESS (optional)----
%% If your preprint is available via ftp or WWW, uncomment
%% the line below by removing the % sign in first column
%% and enter the ftp or WWW address.

%{{\em For preprints via ftp or WWW}: http://www.myaddress } 
%% 
%%             ----CATEGORY------
%% Below give the subject category of the abstract (stellar, solar).
%% If you aren't sure, then give Category: uncertain.  
%% The category will only be used for sorting, and wont be printed     
%% in the newsletter.
{ Category: stellar }
%
%\vspace*{0.3cm}
\end{document}

